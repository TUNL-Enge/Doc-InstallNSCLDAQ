% Created 2015-10-30 Fri 10:32
\documentclass[11pt]{article}
\usepackage[utf8]{inputenc}
\usepackage[T1]{fontenc}
\usepackage{fixltx2e}
\usepackage{graphicx}
\usepackage{longtable}
\usepackage{float}
\usepackage{wrapfig}
\usepackage{rotating}
\usepackage[normalem]{ulem}
\usepackage{amsmath}
\usepackage{textcomp}
\usepackage{marvosym}
\usepackage{wasysym}
\usepackage{amssymb}
\usepackage{hyperref}
\tolerance=1000
\usepackage{fullpage} \usepackage{times} \usepackage{enumitem} \setlist{nolistsep,leftmargin=*}
\author{Richard Longland and John Kelley}
\date{2014-10-03 Fri}
\title{To Install NSCL DAQ}
\hypersetup{
  pdfkeywords={},
  pdfsubject={},
  pdfcreator={Emacs 24.3.1 (Org mode 8.2.4)}}
\begin{document}

\maketitle


\section*{Prerequisites}
\label{sec-1}
\subsection*{General}
\label{sec-1-1}
\begin{itemize}
\item build-essential
\item openssh-server
\item magit
\item libgsl0-dev
\item subversion
\end{itemize}

\subsection*{nscldaq}
\label{sec-1-2}
\begin{itemize}
\item swig2.0 (swig in debian)
\item gengetopt
\item tcl8.5-dev         \# defaults to tcl8.6 in ubuntu
\item tk8.5-dev          \# defaults to tk8.6 in ubuntu
\item libtk    (tklib in debian)     \# these give runtime errors if not present
\item libtcl   (tcllib in debian)
\item libcppunit-dev  \# no errors in configure about this but it's needed!
\item ftpd
\item rsh-server
\item rsh-client
\item libusb-dev
\item tcllib
\item tklib
\end{itemize}
\subsection*{spectcl}
\label{sec-1-3}
\begin{itemize}
\item libmotif-dev  (liblestif2-dev in debian)
\item imagemagick
\item libgd-dev  (libgd2-xpm-dev in debian)
\item byacc           \# build error about yacc
\item flex
\item bison
\item gri
\item libtk-img-dev
\item itcl3-dev
\item itk3-dev
\item iwidgets4
\item bwidget
\item blt-dev
\item libxt-dev
\end{itemize}
\section*{General}
\label{sec-2}
\begin{itemize}
\item Install in \verb~/usr/opt/~
\item \textbf{DO NOT TRY ANYTHING ELSE!} There are some hard coded references
in the code to this directory so it's just not worth your
trouble\ldots{}
\item Build everything as root
\begin{itemize}
\item \verb~sudo su -~
\item \verb~cd /usr/src/~
\item \verb~mkdir NSCLDAQ~
\item \verb~mkdir SpecTcl~
\end{itemize}
\end{itemize}
\section*{For NSCLDAQ}
\label{sec-3}
\begin{itemize}
\item \url{http://sourceforge.net/projects/nscldaq/}
\item I'm using Version 10.2.108
\item By the way, if you screw up in here, make sure you \verb~make clean~
and remove any installed files before trying again!
\item Download to \verb~NSCLDAQ~ directory and untar (\verb~tar -xzvf~
    \verb~nscldaq-10.2-108.tar.gz~)
\item \verb~cd nscldaq-10.2-108~
\item Compile
\begin{itemize}
\item \verb~./configure --prefix=/usr/opt/nscldaq/nscldaq-10.2-108 --enable-usb~
\item \verb~make~
\item \verb~make install~
\item \verb~ln -s /usr/opt/nscldaq /usr/opt/daq~
\end{itemize}
\end{itemize}
\section*{For SpecTcl}
\label{sec-4}
\begin{itemize}
\item \url{http://sourceforge.net/projects/nsclspectcl/}
\item I'm using Version SpecTcl-3.3-016
\item Download to \verb~NSCLDAQ~ directory and untar (\verb~tar -xzvf~
    \verb~SpecTcl-3.3-016.tar.gz~)
\item \verb~cd SpecTcl-3.3-016~
\item \verb~./configure --prefix=/usr/opt/spectcl/SpecTcl-3.3-016~\\
    \verb~--with-tcl-libdir=/usr/lib/x86_64-linux-gnu/~
\item \verb~make~
\item Touch some files to avoid errors in documentation
\begin{itemize}
\item \verb~touch ccusb/dummy.html~
\item \verb~touch vmusb/dummy.html~
\end{itemize}
\item \verb~make install~
\end{itemize}
\section*{Set up some "current" links}
\label{sec-5}
\begin{itemize}
\item \verb~ln -s /usr/opt/nscldaq/nscldaq-10.2-108 /usr/opt/nscldaq/current~
\item \verb~ln -s /usr/opt/spectcl/SpecTcl-3.3-016/ /usr/opt/spectcl/current~
\end{itemize}

\section*{Post-install}
\label{sec-6}
\begin{itemize}
\item Log out of root
\begin{itemize}
\item \verb~exit~
\item \verb~cd~
\end{itemize}
\item Do "the ssh trick"
\begin{itemize}
\item \verb~ssh localhost~
\item Answer "yes"
\item Enter password to log in, then \verb~exit~ to log out
\item \verb~ssh-keygen~ and don't use a password (choose all defaults)
\item \verb,cat ~/.ssh/id_rsa.pub >> ~/.ssh/authorized_keys,
\end{itemize}
\item Find the nscldaq file in the source directory
\begin{itemize}
\item \verb~mkdir -p /home/daq/Live~
\item \verb,cp /usr/src/NSCLDAQ/nscldaq-10.2-108/nscldaq ~/NSCLDAQ,
\item Edit the nscldaq file (This has all been done and the file is in
      \verb~/home/daq/NSCLDAQ/PostInstallFiles~)
\begin{itemize}
\item Make \verb~DAQHOME=/usr/opt/nscldaq/current~
\item Fix the bash script to make \verb~==~ into \verb~=~
\item Make \verb~PIDFILEDIR=/home/daq/Live~
\item Make sure \textasciitilde{}PORTMGRSWITCHES=""\textasciitilde{}
\end{itemize}
\item Make it executable: \verb~chmod u+x nscldaq~
\item \verb~sudo ./nscldaq start~
\item \verb~sudo ./nscldaq status~
\item \verb~sudo ./nscldaq stop~
\item \verb~sudo ./nscldaq status~
\end{itemize}
\item Copy this file into \verb~/etc/init.d/~
\item Link in runlevels so that it starts on boot\\
    \verb~sudo update-rc.d nscldaq defaults~
\item \verb~ls /etc/rc2.d/~ (You should see it in there somewhere)
\item Reboot and check if it's running\\
    \verb~sudo /etc/init.d/nscldaq status~\\
    or\\
    \verb~ps aux | grep DaqPortManager~\\
    \verb~ps aux | grep RingMaster~
\end{itemize}

\section*{Environment}
\label{sec-7}
\begin{itemize}
\item In .bashrc put the following (my version is in PostInstallFiles)
\begin{itemize}
\item \verb~export HOMEDIR=$HOME~
\item \verb~export NSCLBASE=$HOME~
\item \verb~export DISTDIR=/usr/opt/applications~
\item \verb~export BinDir=$HOMEDIR/bin~
\item \verb~export DAQHOST=localhost~
\item \verb~export SSHTARGET=localhost~
\item \verb~export INSTROOT=/usr/opt/nscldaq/current~
\item \textasciitilde{}export TCLLIBPATH="\$TCLLIBPATH \$INSTROOT/lib \$INSTROOT/Scripts \$INSTROOT/TclLibs"\textasciitilde{}
\item \verb,export PATH=$PATH:/usr/opt/nscldaq/current/bin:/usr/opt/spectcl/current/bin:~/bin,
\end{itemize}
\item There are some files needed in \verb,~/bin,. I'll put these in \verb~/home/daq/NSCLDAQ/PostInstallFiles/bin~
\begin{itemize}
\item Menu
\item startCfd
\item startReadout
\item startScaler
\item startSpecTcl
\end{itemize}
\item Put the Menu application in \verb~/usr/opt/applications~.\\
    I've put it in \verb~/home/daq/NSCLDAQ/PostInstallFiles/menu~
\begin{itemize}
\item \verb~sudo mkdir /usr/opt/applications~
\item \verb~sudo cp -r /home/daq/NSCLDAQ/PostInstallFiles/menu /usr/opt/applications/~
\end{itemize}
\item Make an event directory and link it
\begin{itemize}
\item \verb,mkdir ~/events,
\item \verb,ln -s ~/events ~/stagearea,
\end{itemize}
\end{itemize}

\section*{Bin files}
\label{sec-8}
These files are all found in \verb~/home/daq/NSCLDAQ/PostInstallFiles/bin/~
\begin{itemize}
\item Menu
\begin{itemize}
\item A simple bash script to run the menu application
\item \verb~wish $DISTDIR/menu/Menu.ui.tcl &~
\end{itemize}
\item startReadout
\begin{itemize}
\item Make sure this points at
\item \verb~/usr/opt/nscldaq/current/bin/ReadoutShell~
\item with \verb~-host=localhost~
\item and \verb~-path=/usr/opt/nscldaq/current/bin/VMUSBReadout~
\end{itemize}
\item startScalers
\begin{itemize}
\item Make sure this points at
\item \verb~/usr/opt/nscldaq/current/bin/ScalerDisplay~
\item Read the settings from \verb~/config/scalerConfig.tcl~
\end{itemize}
\item startSpecTcl
\begin{itemize}
\item This one points at SpecTcl run file
\item cd into the script directory\\
      \verb,cd ~/config,
\item \verb~exec /usr/opt/spectcl/current/bin/VMUSBSpecTcl </dev/null &~
\end{itemize}
\end{itemize}

\section*{Setup Experiment}
\label{sec-9}
\begin{itemize}
\item The setup is in \verb,~/config,
\item Copy this from  \verb~/home/daq/NSCLDAQ/PostInstallFiles/config/~
\item Do the same for the \verb~spectcl~ directory
\begin{itemize}
\item \verb~cp /home/daq/NSCLDAQ/PostInstallFiles/spectcl /home/daq/~
\end{itemize}
\end{itemize}
\section*{Running with USB}
\label{sec-10}
\begin{itemize}
\item Users need access to the USB device. If you get an error that
looks like\\
    \verb~CTheApplication caught a string exception: usb_get_string_simple failed in CVMUSBusb::serialNo~\\
\end{itemize}
It's probably because the user does not have USB access.
\begin{itemize}
\item First check that the VM-USB card is found by:
\begin{itemize}
\item Run \verb~tail -f /var/log/syslog~
\item Unplug and replug the USB cable
\end{itemize}
\item Some udev rules need to be set
\begin{itemize}
\item Edit \verb~/etc/udev/rules.d/90-usb.rules~
\begin{verbatim}
SUBSYSTEM=="usb", ENV{DEVTYPE}=="usb_device",   MODE="0666"
\end{verbatim}
\item \textbf{NOTE:} This is slightly different from the \verb~usb_device~
subsystem used in previous versions
\item This will allow users to read and write to the usb device
\end{itemize}
\item \textbf{If this doesn't work}
\begin{itemize}
\item First try changing 90 to 95 in the filename above. No need to
reboot, just unplug and replug the USB cable
\item here are some useful testing utilities
\item Find the device (not simply \verb~/dev/usb0~ as in old linux kernels)
\begin{itemize}
\item In Ubuntu, do the following. In debian, you need to figure 
out which device to use some other way!
\item \verb~less /var/log/udev~
\item Look for VM-USB
\item eg. \verb~DEVNAME=/dev/bus/usb/002/004~
\item Use this \verb~DEVNAME~ in the commands below
\end{itemize}
\item Read all of the attributes of this device with\\
      \verb~udevadm info -a -n /dev/bus/usb/002/004~
\item Test the udev rules as you edit them with\\
      \verb~udevadm test $(udevadm info -q path -n /dev/bus/usb/002/004) 2>&1~
\item You should see the \verb~/etc/udev/rules.d/90-usb.rules~ get sourced
and the permissions of the device get set to "0666"
\end{itemize}
\end{itemize}
% Emacs 24.3.1 (Org mode 8.2.4)
\end{document}